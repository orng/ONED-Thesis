\chapter{Introduction}
Iceland and Sweden are two countries that despite being very different in many ways also have a lot in common. As two members of the Nordic countries they share much in terms of history, culture, language and more. Over the past centuries they have collaborated closely in many matters and as nordic countries they share many of the trait of a common market; citizens of either country require no special permission to travel to, move to or work in the other country~\cite{norden}.

In this report we will examine the hiring practices within the software industries in these two countries and compare them to each other in terms of hiring processes, job benefits, salary negotiations, labour unions and more. In particular we will focus on processes involving computer programmers and software developers.

\chapter{The job markets}
For several years the demand for software developers in Iceland has been very high. One might in fact go so far as to call it a shortage. Programmers with degrees in computer science, software engineering or related fields such as mathematics or other engineering fields typically have jobs waiting when they finish their Bachelor's degrees and experienced developers can typically have their pick of jobs with fairly little competition.
In fact, only a relatively small subset of students continue their studies onto Master's level, and those who do often spend a few years working in the industry first.

Although the demand for software developers is also high in Sweden~\cite{Spotify}, it is more common for Swedish students than Icelandic ones to complete their Master's level studies prior to entering the job market. Furthermore, Swedish students typically conduct work for companies within the software development sector as part of their Master's thesis work and frequently get job offers as a result. Although some students in Iceland do conduct Bachelor or Master thesis studies in collaboration with software development companies it is a lot less common than in Sweden.

In both countries the hiring process varies between companies but most companies follow the same basic procedure: Job openings are advertised on-line, in newspapers or both. Applications usually include CVs and once the deadline for applications has expired, applicants are screened based on whether or not they fulfil the requirements detailed in the job posting, with regards to education and experience. Applicants who pass the initial CV screening are typically interviewed at least once and in some cases asked to perform tasks to prove their skills outside of interviews.

Although large international companies which attract a lot of attention and applicants seem to typically require more thorough interviewing and testing of applicants, the size of companies is not a deciding factor. In the five different times the author has searched for programming jobs he has had to solve programming tasks prior to getting jobs at companies with employee numbers ranging from 10 to 4.000. One of these companies, CCP Games had the author take a 5 hour test at home prior to being invited to a series of three interviews in which he had to solve problems on a whiteboard. In contrast, flight search engine company Dohop tasked the author with solving a 4 hour programming task after a fairly simple and straight forward job interview, immediately offering him a job upon completion.

\section{Referrals}
Referrals from former employees, teachers or current employees at the hiring company are often deciding factors in the hiring process. Since Iceland is such a small country it is not at all uncommon for there to be at least someone within the hiring company with ties to an applicant. Perhaps the hiring manager grew up with the applicant's father, or plays football with his former boss during lunchtime on Fridays, or perhaps the applicant went to school with the lead programmer of the team he would be joining. Although very few companies, if any, have referral bonuses for employees it is not uncommon for applicants to have heard of the companies or the vacant positions through friends or family members. Although referrals, contacts and networking are important factors in Sweden, the IT scene is larger overall and especially in larger cities and therefore people and companies are less connected. 

In any case, once an applicant has passed through enough screenings to be considered a likely candidate for a position, the hiring manager will typically call at least one of his listed referrals and in some cases his former employees although they have not been listed as referrals. 

\section{Relocation}
With the vast majority of people in Iceland living in the Reykjavik region~\cite{hagstofa}, most software developing companies are located there as well and it is almost always assumed that people take care of housing and logistics themselves. The typical exception to this rule is when foreign talent is acquired, however few Icelandic companies operate at the level or scale where they actively search for international talents. 

Relocating for jobs is more common in Sweden. Both because Sweden is a much larger country and software developing companies and offices are scattered across several different widespread cities but also because Sweden houses larger and more specialized companies which require world-grade specialists. Therefore acquisition of foreign talent is more common in Sweden.

\section{Salaries and benefits}
Salaries are similar between the two countries, with Icelandic salaries being slightly higher. In 2015 the average base salary for programmers in Iceland being around 600.000 ISK~\cite{VR} and base salaries in Sweden appear to be somewhere between 30.000\cite{lon1} and 40.000 SEK~\cite{lon2} which translates to 390.000 - 520.000 ISK at the current exchange rate~\cite{arion}. However, the cost of living in Iceland is high~\cite{grapevine} so the difference is not that large.

Both Sweden and Iceland have good government provided social benefits such as healthcare and pension and in both countries companies are required to provide similar benefits to employees regarding sick pay, parental leave, vacation and insurance. Similarly, required office hours in both countries is typically 40 hours per week.

\chapter{Conclusion}
Overall, the job market for programmers is similar between Iceland and Sweden and as such companies in either country follow similar recruiting procedures. The biggest differences stem from the difference in size of the two countries: Iceland with its small size provides an environment where "everyone knows everyone" and contacts play a larger roll in the recruiting process whereas Sweden with its larger size and larger companies makes it more common for people to relocate for jobs and thus more companies have to play large parts in helping with the logistics involved in such transfers. In both countries the demand for programmers is high and as such programmers which desire to work in either country can probably afford to be optimistic about the future.

%\chapter*{Abstract}
%\label{chapter:abstract}
\begin{abstract}
In today's society the amount of information available is vast. The internet, one of the largest sources of information available today, makes finding information easy. Although large parts of the internet consist of images and videos of cats doing silly things there are still numerous sources with structured text information stream, such as news. However, the number of available news streams is so wast that no man is capable of digesting them all. Moreover, the various news sources tend to re-iterate upon events previously reported by themselves or other sources. It can therefore become tedious to try and stay on top of the news. 

\emph{On-line New Event Detection} can be used to monitor multiple such news streams and determine when a new newsworthy event that might further warrant attention occurs, thus helping the user to not miss out on important information while helping him sift through the vast amount of information available.

In this thesis we present a novel approach for On-line New Event Detection based on computing minimal new sets of small sizes from new stream documents with regards to previously seen documents. We present an algorithm that outputs sets of words likely to be indicative of new events which users easily can browse through and detail several different approaches to select the sets most likely to be informative of new events. In addition we examine the practicality of our approaches and examine the runtime and memory complexities of our algorithms in relation to the most prominent current techniques.
\end{abstract}

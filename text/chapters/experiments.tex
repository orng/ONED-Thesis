\chapter{Experiments}
\label{chapter:experiments}

We implemented the concepts discussed in chapter~\ref{chapter:method} using the Python programming language. We then used these implementations to test the effectiveness of the various methods on several different datasets. In this chapter we discuss our implementation in some detail as well as how the tests were carried out and what data was used for testing.

\section{Implementation}
For our implementation we chose Python because it supports many high-level concepts such as lambda functions, list comprehensions etc. as well as for its extensive standard library and excellent availability of good third-party libraries such as the Natural Language Tool Kit (NLTK)~\cite{nltk}. The most important files are included in \cref{appendix:source} but the full source used in this project can be found on GitHub\footnote{\url{https://github.com/orng/ONED-Thesis}}.

\subsection{Pre-processing}
The NLTK provided us with most of the parts needed for preprocessing: we used the built-in word-tokenizer for splitting whole-string articles into lists of words,  stopwords from the English corpus for stopword-removal and the porter2 stemmer for fast and simple stemming of words. Python's string module provided us with a list of punctuation characters which we used to remove punctuation. Our pre-processing step first removed any punctuation character, then tokenized the input-string, removed all the stopwords and then stemmed each of the tokens.

\subsection{Enumeration}
We implemented the algorithm described in section~\ref{method:enumeration}. Our implementation used sets of strings (or Python's frozensets in cases where the sets had to be hashed) to represent the bags of words to enumerate, a list of bags (list of sets of strings) to store previously seen bags and a dictionary (hashmap) with frozensets as keys and integers as values for mapping previously seen minimal new sets to the number of the bag in which they first appeared. 

As discussed by Damaschke~\cite{damaschke2015pairs} the existence of small minimal-new sets is likely to prove more informative than large sets. In order to simplify results we therefore chose to limit the enumerations to only consider subsets of size two or smaller.

\section{Experiments}
We conducted a series of tests on various data sources using the base-approach to enumeration as well as the different extensions described in \cref{chapter:method}. We created several configurations, each using one or more of the aforementioned approaches. Each configuration was tested on four different sets of articles and the result of each processed article was stored into a text file. Additionally, for any word found in a minimal new set of size two we printed the amount of such minimal new sets that the word occurred in. The configurations we used in our experiments were as follows:

%NOTE TO SELF: renaming configurations here, this will no longer match the result files.
% Mapping (old->new): 3->2, 5->3, 6->4

\emph{Configuration-1}: The base algorithm without any further analysis or filtering of the input and output sets beyond the basic pre-processing described in section~\ref{method:preprocessing}. Additionally, we conducted experiments with sorting new pairs in descending order based on the minimum distance between the words in the pairs or the combined number of minimal-new sets that the words appeared in. In the latter case we also outputted the number of minimal-new sets the words appeared in.

\emph{Configuration-2}: Output filtering where any minimal-set containing at least one of the $k$ highest scoring words was included in the output, for positive integer numbers of $k$. 

\emph{Configuration-3}: Filtering using the same technique as configuration-2, except filtering was applied prior to enumeration.

\emph{Configuration-4}: Processing articles by splitting them into sub-bags where each sub-bag was a sentence.


\subsection{Data}
All of the data that we used for testing was real data obtained by crawling various sources from the internet. They contain articles of various types and lengths since one of our goals was to identify to some extent for what type of input the approaches were successful. The crawled articles were stored along with their dates of publication so that they could later be consumed in the same order that they would have arrived had they been processed by an on-line monitoring system. We used the following four different datasets for testing:

\emph{D1}: A collection of very short summary articles and headlines collected from articles regarding the 2016 presidential election in the USA, gathered from PBS Newshour\footnote{\url{http://www.pbs.org/newshour/tag/vote-2016}}, Reuters\footnote{\url{http://www.reuters.com/politics/election2016}} and CBS News\footnote{\url{http://www.cbsnews.com/election-2016/}}. The data was gathered on 04.04.2016 containing 300 stories published on dates ranging from 03.03.2016 to 04.04.2016. This dataset served to test our approach on short stories within a narrow topic. Our hope was that short articles within a shared topic would provide good results, since high word density within a shared topic combined with the low overall word count of each article would decrease the amount of new words found and instead yield interesting new pairs.

\emph{D2}: Timelines of president Barack Obama's years of presidency, '09, '10 and '11, taken from Wikipedia\footnote{\url{https://en.wikipedia.org/wiki/Timeline_of_the_presidency_of_Barack_Obama}}. Each point of the timeline can be regarded as a very short article and in this case is very likely to refer to an event. The data was gathered on 26.04.2016 and contained 419 stories in total.

\emph{D3}: A chronological account of the events of the Second World War taken from world-war-2.info\footnote{\url{http://world-war-2.info/history/index.php}} where we treated each paragraph as a separate article. The point of this dataset was to test our approaches on slightly longer and less densely focused articles within a single topic. The data was gathered on 21.03.2016 and contained 495 stories in total.

\emph{D4}: Real life full-length news articles taken from Reuters\footnote{\url{http://www.reuters.com/news}}. The data was gathered on 03.08.2016 and includes 5227 articles dating from 04.11.2016 to 03.08.2016. The purpose of this dataset was to simulate real-life usage of an ONED system on a large set of data.

\subsection{Evaluation}
Evaluation of the results was done through manual evaluation and comparison of the results outputted by the experiments. In particular, we were interested in finding methods that prioritized minimal new sets that captured the main points of the input articles. 


In chapter~\ref{chapter:results} we present the results of our tests and in \cref{chapter:discussion} we discuss and analyse those results.

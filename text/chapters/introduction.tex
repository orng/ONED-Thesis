\chapter{Introduction}
\label{chapter:introduction}
Over the last few decades the amount of electronically available information has grown rapidly. The daily amount of information that is uploaded to sources such as the internet is so vast that it has become impossible for any human to keep process it all and keeping track of important events can be overwhelming. Within several domains, such as stock trading, it can be of vital importance to be able to identify important events as soon as they occur and thus the availability of fast and reliable computer systems that aid in the detection of such events can be of great benefit. 

\emph{On-line New Event Detection} (ONED) is the process of monitoring text information streams and detecting stories that report about new events. For instance, a story reporting a large oil leakage in the Pacific Ocean should be identified as a new story the first time that the event is reported while consecutive stories further iterating upon the event or stories discussing the environmental effects of the spill should be identified as not containing new events. 

In this thesis we present a new approach to ONED built upon Damaschke's concept~\cite{damaschke2015pairs} of identifying minimal new sets of words from new articles and using those as basis for determining whether or not the text refers to a new event. The intuition, as described by Damaschke, is that when a new event occurs, like the death of a celebrity, we are likely to find words together within the same article that have never occurred before. Moreover we are likely to find small sets of words that have never previously occurred together. For instance, the first report of the recent passing of the artist Prince would likely have been the first time the words ``Prince'' and ``dies'' appeared together. By finding and filtering these minimal new sets to only include sets containing user specified keywords we are able to present likely new events relating to the given keywords.

We will start this thesis off by discussing the background ONED and related works on the topic in chapter~\ref{chapter:background}. We then go through the goals and details of our method in chapter~\ref{chapter:method}. In chapter~\ref{chapter:results} we go present results of tests on our method. Finally, in chapter~\ref{chapter:discussion} we discuss our findings and present suggestions for future work for improvement.

\chapter{Introduction}
\label{chapter:introduction}
In today's society the amount of information available is vast. The internet, one of the largest sources of information available today, makes finding information easy. Although large parts of the internet consist of images and videos of cats doing silly things there are still numerous sources with structured text information streams, such as news. However, the number of available news streams is so wast that no man is capable of digesting them all. Moreover, the various news sources tend to re-iterate upon events previously reported by themselves or other sources. It can therefore become tedious to try and stay on top of the news. 
Within several domains, such as stock trading, it can be of vital importance to be able to identify important events as soon as they occur and thus the availability of fast and reliable computer systems that aid in the detection of such events can be of great benefit. 

%Over the last few decades the amount of electronically available information has grown rapidly. The daily amount of information that is uploaded to sources such as the internet is so vast that it has become impossible for any human to process it all and keeping track of important events can be overwhelming. 

\emph{On-line New Event Detection} (ONED) is the process of monitoring text information streams and detecting stories that report about new events. For instance, a story reporting a large oil leakage in the Pacific Ocean should be identified as a new story the first time that the event is reported, while consecutive stories further iterating upon the event or stories discussing the environmental effects of the spill should be identified as not containing new events. 

In this thesis we present a new approach to ONED, built upon the concept of identifying minimal new sets of words from new articles and using those as basis for determining whether or not the text refers to a new event. This concept has previously been proposed by Damaschke~\cite{damaschke2015pairs} and Wurzer et al.~\cite{wurzer2015kterm}. The intuition is that when a new event occurs, like the death of a celebrity, we are likely to find words together within the same article that have never occurred before. Moreover we are likely to find small sets of words that have never previously occurred together. For instance, the first report of the recent passing of the artist Prince would likely have been the first time the words ``Prince'' and ``dies'' appeared together. By finding and filtering these minimal new sets to only include informative word combinations based on quantitative criteria we are able to present likely events succinctly which allows a user to browse through them.

We will start this thesis off by discussing the background of ONED and related works on the topic in chapter~\ref{chapter:background}. We then go through the goals and details of our method in chapter~\ref{chapter:method}. In chapter~\ref{chapter:experiments} we explain the details of our implementation used for the testing and the methods we employ for testing. Chapter~\ref{chapter:results} presents the results of the tests. Finally, in chapter~\ref{chapter:discussion} we discuss our findings and present suggestions for future work.
